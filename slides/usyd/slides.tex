% !TeX program = xelatex
% !BIB program = biber
\documentclass[aspectratio=169, 12pt]{beamer}
\usetheme{usyd}

% --- Title Page Meta Data ---
\title{\large Optimal Look-back Horizon for Time Series Forecasting in Federated Learning}
\subtitle{}\vspace*{-1.5em}
\author{\noindent
  Dahao Tang\textsuperscript{1},
  Nan Yang\textsuperscript{1},
  Yanli Li\textsuperscript{1},
  Zhiyu Zhu\textsuperscript{2},
  Zhibo Jin\textsuperscript{2},
  Dong Yuan\textsuperscript{1}
}\vspace*{1.5em}
\institute{\noindent
  \textsuperscript{1}University of Sydney \\
  \textsuperscript{2}University of Technology Sydney \\
}
\date{}
\secondlogo[height=1.8cm]{AAAI26}

% --- Bib File ---
\addbibresource{refs.bib}

\begin{document}

% --- Title Slide ---
\begin{frame}
  \titlepage
\end{frame}

% --- Table of Contents ---
\begin{frame}
  \tableofcontents
\end{frame}

% --- Main ---
\section{Introduction \& Problem Gap}
\begin{frame}{Introduction}
  Selecting an appropriate {\color{usydred}look-back horizon} remains a fundamental challenge in 
  time series forecasting (TSF), particularly in the federated learning (FL) scenarios 
  where data is decentralized, heterogeneous, and often non-independent.
\end{frame}

\begin{frame}{Problem Gap: Horizon Selection in Federated TSF}
    \small
    \textbf{1. Horizon Selection in Time Series Forecasting (TSF)}
    \begin{itemize}
        \item Traditional methods (e.g., ARIMA, LSTMs, Transformers) treat the look-back horizon as 
              a tunable hyperparameter set via heuristics or validation, often leading to overfitting or inefficient data use
              \cite{akaike1974new,box2015time,hochreiter1997long,lim2020temporal,zhou2021informer,woo2023deeptime,koparanov2020lookback}.
    \end{itemize}
    \textbf{2. Intrinsic Representation}
    \begin{itemize}
        \item Shi et al (2024) introduced the idea of an intrinsic space framework for dynamic horizon selection, but assumes centralized, IID data 
        \cite{shi2024scaling}. The framework is lack of step-by-step transformation from time series into the intrinsic space.
    \end{itemize}
    \textbf{3. Federated Learning (FL) for TSF}
    \begin{itemize}
        \item Prior FL research focuses on aggregation algorithms (FedAvg, FedProx) or model architectures 
              but does not address horizon selection, especially under the non-IID client dynamic settings 
              \cite{mcmahan2017communication,li2020federated,perifanis2023federated}.
    \end{itemize}
\end{frame}

\begin{frame}{Research Goal}
  {\color{usydred}Goal:} Find the optimal look-back horizon for time series forecasting in
  the non-IID data setting under the federated learning scenario.
\end{frame}

\section{Solution Pipeline \& Contributions}
\begin{frame}{Solution Pipeline \& Contributions}
  \footnotesize
  We propose a principled theoretical framework for horizon selection in non-IID federated learning:
  \begin{itemize}
    \item \textbf{\color{usydred}Constructive Intrinsic Space via SDG:}
    We introduce a Synthetic Data Generator (SDG) that explicitly models temporal dynamics (autoregression, seasonality, trend) and client heterogeneity.
    This allows us to construct a geometry-preserving intrinsic space and derive a computable Intrinsic Dimension $d_{I,k}(H)$ based on signal saturation.
    \item \textbf{\color{usydred}Federated Loss Decomposition:}
    We establish an FL-based decomposition of predictive loss into Bayesian and Approximation components, 
    each analytically tied to the structural elements of time series data and the look-back horizon. 
    Our analysis shows the fundamental bias–variance trade-off that governs forecasting performance in federated settings.
    \item \textbf{\color{usydred}Provably Optimal Horizons ($H_k^*$ and $H^*_{server}$):}
    We prove that the client-wise total loss is (conditionally) unimodal with respect to the horizon length and identify the smallest sufficient horizon as its global minimiser. 
    This result provides the first rigorous criterion for horizon selection in time series forecasting 
    % and introduces a new design principle for model construction 
    under sample-limited, heterogeneous environments.
  \end{itemize}
\end{frame}

\section{Preliminary}
\begin{frame}{SDG: Modeling Temporal Structures}
  We model the observation $\hat{x}_{f,t,k}$ for feature $f$, time $t$, and client $k$ as:
  \begin{equation}
    \label{eq:sdg}
    \hat{x}_{f,t,k} = 
      \underbrace{\sum_{j=1}^{J} A_{f,j,k} \cdot \sin \left( \frac{2 \pi t}{T_{f,j,k}} + \theta_{f,j,k} \right)}_{\text{Seasonality}} 
    + \underbrace{\sum_{i=1}^{p} \phi_{k,i} \ x_{f,t-i,k}}_{\text{AR Memory}} 
    + \underbrace{\beta_{f,k} \ t}_{\text{Trend}} 
    + \underbrace{\epsilon_{f,t,k}}_{\text{Noise}}
  \end{equation}
  
  \begin{itemize}
    \item \textbf{Seasonality:} Defined by amplitude $A_{f,j,k}$, period $T_{f,j,k}$, 
          and phase $\theta$.
    \item \textbf{AR Memory:} Autoregressive process with client-specific 
          lag coefficients $\phi_{k,i}$.
    \item \textbf{Trend:} Linear component slope $\beta_{f,k}$.
    \item \textbf{Noise:} Gaussian innovation $\epsilon \sim \mathcal{N}(\mu_{f,k}, \sigma_{f,k}^2)$.
  \end{itemize}
\end{frame}

\begin{frame}{Capturing Feature Heterogeneity}
  Real-world FL data exhibits feature skew. We model this via affine transformations:
  \begin{equation}
    x_{f,t,k} = \Lambda_{f,k} \tilde{x}_{f,t,k} + \delta_{f,k}
  \end{equation}
  \textbf{Explanation of Terms:}
  \begin{itemize}
    \item $\Lambda_{f,k}$ (Linear Scale): Controls the variance $\sigma_f^2$ for feature $f$ on client $k$.
    \item $\delta_{f,k}$ (Mean Shift): Adjusts the mean $\mu_f$ for feature $f$ on client $k$.
  \end{itemize}
  \vspace{0.5em}
  \textit{Significance:} This allows the framework to explicitly account for non-IID distributions where 
  clients observe different scales of the same underlying features.
\end{frame}

\begin{frame}{Intrinsic Space Transformation}
  We construct a geometry-aware representation space that captures the essential temporal structure of non-IID time series
  through a transformation grounded in the SDG. The transformation pipeline proceeds in five steps: 
  \begin{enumerate}
    \item Client-wise normalization to remove affine feature skew and align marginal distributions
    \item Window flattening to convert each normalized time-series segment into a fixed-length vector
    \item Global covariance estimation and eigendecomposition to identify dominant axes of variation
    \item Intrinsic dimension estimation based on the SDG and empirical spectrum
    \item Projection into intrinsic space via principal components
  \end{enumerate}
\end{frame}

\begin{frame}{Intrinsic Dimension Estimation}
  We map windows to an intrinsic space. The dimension $d_{I,k}(H)$ approximates the effective degrees of freedom:
  \begin{equation}
    d_{I,k}(H) \approx F \cdot \left( \min\{H, \ell_{\mathrm{AR},k}\} + g_k(H) + 1 \right)
  \end{equation}
  \textbf{Components:}
  \begin{itemize}
    \item $F$: Number of features.
    \item $\ell_{\mathrm{AR},k}$: Effective AR memory length (how far back history matters).
    \item $g_k(H)$: Resolved seasonal complexity.
    \item $+1$: Accounts for the linear trend component.
  \end{itemize}
\end{frame}

\begin{frame}{Detailed Intrinsic Components}
  \textbf{1. Effective AR Memory ($\ell_{\mathrm{AR},k}$):}
  \begin{equation}
    \ell_{\mathrm{AR}, k} = \left\lceil \frac{\ln(1 / (1 - \epsilon))}{- \ln \rho_k} \right\rceil, \ \epsilon\in(0,1)
  \end{equation}
  \begin{itemize}
    \item $\rho_k$: Spectral radius of the AR companion matrix (stability metric).
    \item Represents the time steps needed for impulse response to decay.
  \end{itemize}
  \textbf{2. Seasonal Complexity ($g_k(H)$):}
  \begin{equation}
    g_k(H) = 2 \sum_{j=1}^{J} w_{j,k} \cdot \min \left(1, \frac{H}{T_{j,k}^*} \right)
  \end{equation}
  \begin{itemize}
    \item Measures how many full seasonal cycles fit within horizon $H$.
    \item Saturates when $H$ exceeds the period $T_{j,k}^*$.
  \end{itemize}
\end{frame}

\section{Loss Analysis}
\begin{frame}{Theorem 1: Federated Loss Decomposition}
  The total prediction loss for a predictor $m$ decomposes into two distinct sources of error for 
  both the client and server side:
  \begin{equation}
    L(H,S;m) = \underbrace{L_{\mathrm{Bayes}}(H,S)}_{\text{Irreducible}} + \underbrace{L_{\mathrm{approx}}(H,S;m)}_{\text{Reducible}}
  \end{equation}
  \begin{itemize}
    \item \textbf{Bayesian Loss ($L_{\mathrm{Bayes}}$):} The error of an ideal predictor with full knowledge of 
          the data distribution. It reflects inherent uncertainty (noise).
    \item \textbf{Approximation Loss ($L_{\mathrm{approx}}$):} The excess error due to using a 
          finite-capacity model trained on finite samples, relative to the Bayes-optimal predictor.
  \end{itemize}
\end{frame}

\begin{frame}{Theorem 2: Client-wise Bayesian Loss}
  For client $k$, the irreducible loss is the sum of component-wise errors:
  \begin{equation}
    L_{\mathrm{Bayes}}^{(k)}(H,S) = L_{\mathrm{AR}}^{(k)}(S) + L_{\mathrm{seas}}^{(k)}(H) + L_{\mathrm{trend}}^{(k)}(H)
  \end{equation}
  \textbf{Behaviour with respect to Horizon ($H$):}
  \begin{itemize}
    \item \textbf{Decreases:} As $H$ increases, we resolve more seasonal phases and trend direction.
    \item \textbf{Saturates:} Once $H$ covers the AR memory and seasonal periods, 
          adding more history provides \textbf{zero} additional information gain.
    \item \textbf{Limit:} $\Delta L_{\mathrm{Bayes}}^{(k)}(H) \to 0$ for large $H$.
  \end{itemize}
\end{frame}

\begin{frame}{Theorem 3: Approximation Loss Bound}
  The approximation error is bounded by curvature (bias) and variance terms:
  \begin{equation}
    L_{\mathrm{approx}}^{(k)}(H,S; m) \lesssim 
    \underbrace{\Big( K_2^2\, d_{I,k}(H)^2 \Big)^{\frac{d_{I,k}(H)}{4 + d_{I,k}(H)}}}_{\text{Curvature / Bias Term}} 
    + \underbrace{\Big( \frac{d_{I,k}(H)\, H}{D_k} \Big)^{\frac{4}{4 + d_{I,k}(H)}}}_{\text{Variance / Finite Sample Term}}
  \end{equation}
  \textbf{Why does this increase with $H$?}
  \begin{enumerate}
    \item \textbf{Intrinsic Dimension ($d_{I,k}$):} Grows with $H$, making the function harder to learn (Curse of Dimensionality).
    \item \textbf{Effective Samples ($D_k/H$):} As window length $H$ grows, the number of independent samples in a fixed dataset decreases, increasing variance.
  \end{enumerate}
\end{frame}

\section{Optimal Horizon Selection}
\begin{frame}{The Fundamental Trade-off}
  The total loss $L^{(k)}(H)$ is (conditionally) minimised at the smallest sufficient horizon $H^*$. 
  % and is achieved at the smallest sufficient horizon ($H_k^*$),
  \begin{itemize}
    \item \textbf{Small $H$:} High Bayesian Loss (Underfitting the dynamics).
    \item \textbf{Large $H$:} High Approximation Loss (Overfitting / Insufficient samples).
  \end{itemize}
  \begin{block}{Smallest Sufficient Horizon ($H_k^*$)}
    Defined as the smallest $H$ where Bayesian loss improvements saturate within a tolerance $\delta$:
    \begin{equation}
      H_k^*(\delta) := \min\{H : |\Delta L_{\mathrm{Bayes}}^{(k)}(H)| \le \delta\}
    \end{equation}
  \end{block}
  % defined as the smallest horizon where Bayes improvements fall below $\delta$:
  % \begin{equation}
  %   H_k^*(\delta) := \min\{H \in \mathbb{N} : |\Delta L^{(k)}_{\text{Bayes}}(H)| \le \delta\}.
  % \end{equation}

  \textbf{Theorem 4 (Margin Conditions).} If there exists $\delta>0$ such that
  \begin{equation}
  \Delta L^{(k)}_{\text{Bayes}}(H) \le -\delta ,\ \forall H < H_k^*(\delta),
  \qquad
  \Delta L^{(k)}_{\text{approx}}(H) \ge \delta ,\ \forall H \ge H_k^*(\delta),
  \end{equation}
  then the total loss $L^{(k)}(H)$ is \textbf{unimodal} and is \textbf{minimized at} $H_k^*(\delta)$ (up to integer ties).

\end{frame}

\begin{frame}{Practical Selection: Seasonal Coverage}
  We can link the tolerance $\delta$ to interpretable seasonal coverage:
  \begin{equation}
    H_k^*(\delta) = \max\{\ell_{\mathrm{AR},k},\ T_k^{(\tau)}\}
  \end{equation}
  \textbf{Interpretation:}
  \begin{itemize}
    \item The optimal horizon is simply the maximum of the \textbf{effective AR memory} and the \textbf{seasonal period} required to capture $(1-\tau)$ of the signal energy.
    \item This provides a theoretically grounded, calculable target for each client.
  \end{itemize}
\end{frame}

\begin{frame}{Global Horizon}
  In Federated Learning, we need a single global horizon $H_{\text{server}}$ despite heterogeneous local optima. We use a weighted trimmed mean:
  \begin{equation}
    H_{\mathrm{server}}^* = \operatorname{TrimMean}_{\alpha}\big(\{H_k^*(\delta)\}_{k=1}^K ; \{w_k\}_{k=1}^K\big)
  \end{equation}
  \begin{itemize}
    \item \textbf{Weights $w_k$:} Proportional to client data size ($n_k$).
    \item \textbf{Trimmed Mean:} Discards the top/bottom $\alpha$-fraction of extreme horizons.
    \item \textbf{Benefit:} Robustness. Prevents a single client with an extremely long seasonality (requiring huge $H$) from degrading the sample efficiency for all other clients.
  \end{itemize}
\end{frame}

\begin{frame}{Limitations \& Future Work}
  \textbf{\color{usydred}Limitations:}
  \begin{itemize}
    \item Assumes a structured SDG, which may not capture complex nonlinear or regime-switching real-world dynamics.
    \item Validated on limited real-world settings, so the generality of the optimal horizon theory across diverse federated datasets remains untested.
  \end{itemize}
  \textbf{\color{usydred}Future work:}
  \begin{itemize}
    \item Extend the SDG and theory to handle nonlinear, non-stationary, or multivariate temporal behaviors beyond the current additive model.
    \item Validate the framework across broader and more challenging federated forecasting benchmarks, including irregular sampling and strong distribution drift.
  \end{itemize}
\end{frame}

\section{Conclusion}
\begin{frame}{Conclusion}
  % \item \textbf{Novel Framework:} 
  This work establishes the first theoretically grounded criterion for adaptive horizon selection, 
  offering practical guidance for model design and benchmarking in decentralized, heterogeneous environments.
  \begin{itemize}
    \small
    \item \textbf{\color{usydred}Novel Framework:} This work introduces a principled approach for 
          selecting input horizons in non-IID federated time series using a Synthetic Data Generator (SDG).
    \item \textbf{\color{usydred}Federated Loss Decomposition:} We show that the decomposition of the forecasting loss into 
          Bayesian error (decreases with horizon) and approximation error (increases with horizon) works under the FL settings.
    \item \textbf{\color{usydred}Optimal Horizon:} We prove that the client-wise total loss is (conditionally) minimised at the smallest sufficient horizon $H^*$, 
          balancing structure identification against the curse of dimensionality.
          We also propose a robust aggregation mechanism to identify a single, 
          effective horizon across heterogeneous clients.
  \end{itemize}
\end{frame}

\begin{frame}
  \color{usydred}
  \centering
  \Huge \textbf{Thank You} \\
  \vspace{1cm}
  \large \textbf{Questions?}
\end{frame}

\section{Reference}
\begin{frame}[allowframebreaks]
  \frametitle{References}
  \printbibliography
\end{frame}

\end{document}